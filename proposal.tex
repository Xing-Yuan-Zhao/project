% Options for packages loaded elsewhere
\PassOptionsToPackage{unicode}{hyperref}
\PassOptionsToPackage{hyphens}{url}
%
\documentclass[
]{article}
\usepackage{amsmath,amssymb}
\usepackage{iftex}
\ifPDFTeX
  \usepackage[T1]{fontenc}
  \usepackage[utf8]{inputenc}
  \usepackage{textcomp} % provide euro and other symbols
\else % if luatex or xetex
  \usepackage{unicode-math} % this also loads fontspec
  \defaultfontfeatures{Scale=MatchLowercase}
  \defaultfontfeatures[\rmfamily]{Ligatures=TeX,Scale=1}
\fi
\usepackage{lmodern}
\ifPDFTeX\else
  % xetex/luatex font selection
\fi
% Use upquote if available, for straight quotes in verbatim environments
\IfFileExists{upquote.sty}{\usepackage{upquote}}{}
\IfFileExists{microtype.sty}{% use microtype if available
  \usepackage[]{microtype}
  \UseMicrotypeSet[protrusion]{basicmath} % disable protrusion for tt fonts
}{}
\makeatletter
\@ifundefined{KOMAClassName}{% if non-KOMA class
  \IfFileExists{parskip.sty}{%
    \usepackage{parskip}
  }{% else
    \setlength{\parindent}{0pt}
    \setlength{\parskip}{6pt plus 2pt minus 1pt}}
}{% if KOMA class
  \KOMAoptions{parskip=half}}
\makeatother
\usepackage{xcolor}
\usepackage[margin=1in]{geometry}
\usepackage{graphicx}
\makeatletter
\def\maxwidth{\ifdim\Gin@nat@width>\linewidth\linewidth\else\Gin@nat@width\fi}
\def\maxheight{\ifdim\Gin@nat@height>\textheight\textheight\else\Gin@nat@height\fi}
\makeatother
% Scale images if necessary, so that they will not overflow the page
% margins by default, and it is still possible to overwrite the defaults
% using explicit options in \includegraphics[width, height, ...]{}
\setkeys{Gin}{width=\maxwidth,height=\maxheight,keepaspectratio}
% Set default figure placement to htbp
\makeatletter
\def\fps@figure{htbp}
\makeatother
\setlength{\emergencystretch}{3em} % prevent overfull lines
\providecommand{\tightlist}{%
  \setlength{\itemsep}{0pt}\setlength{\parskip}{0pt}}
\setcounter{secnumdepth}{-\maxdimen} % remove section numbering
\ifLuaTeX
  \usepackage{selnolig}  % disable illegal ligatures
\fi
\IfFileExists{bookmark.sty}{\usepackage{bookmark}}{\usepackage{hyperref}}
\IfFileExists{xurl.sty}{\usepackage{xurl}}{} % add URL line breaks if available
\urlstyle{same}
\hypersetup{
  pdftitle={proposal},
  hidelinks,
  pdfcreator={LaTeX via pandoc}}

\title{proposal}
\author{}
\date{\vspace{-2.5em}2023-06-23}

\begin{document}
\maketitle

\hypertarget{topic-of-interest-and-motivations}{%
\subsubsection{Topic of Interest and
motivations}\label{topic-of-interest-and-motivations}}

Conventional wisdom holds that right-wingers should be hawkish and are
supposed to tend to support a more aggressive foreign policy compared to
liberals, however, this does not seem to be the case for American
Republicans in the Russia's invasion of Ukraine scenario, according to
the following accounts. 19 GOP lawmakers urged Biden to stop sending
``unrestrained'' aid and weapons to Ukraine (The Hill. 2023), and Donald
Trump has also claimed to try to broker peace in Ukraine (Business
Insider. 2022; Business Insider. 2023). On the other hand, Boris
Johnson, the former Prime Minister and the leader of the Conservative
Party, who has been supporting the Ukrainian defenders from the
beginning, tried to convince Trump to support Ukraine(The Guardian.
2023), and the Italian leader Meloni, who leads a far-right party, is
also willing to risk unpopularity to support Ukraine.(Reuters. 2023)
Poland, led by United Right, also has been supported Ukraine almost
since day one and pleged a lot by its GDP size.(The Economist. 2022) The
most interest part is that, the Trump supporters themselves were once
considered more hawkish(Sides et al.~2018), but now they want world
peace.

\hypertarget{research-question}{%
\subsubsection{Research Question}\label{research-question}}

So now we can see a very counter-intuitive problem: \textbf{What is
discouraging American conservatives from taking a more hawkish stance in
support of Ukraine?}

\hypertarget{hypothesis}{%
\subsubsection{Hypothesis}\label{hypothesis}}

H1: Conservatives in the US are built differently. They hold different
sets of beliefs than other conservatives, such as isolationism.(Clarke
and Ricketts 2017; Rolf 2021; Cha 2016; Clark et al.~2016)

H2: They genuinely believe that supporting Ukraine will lead to WW3 or
nuclear war, and they want to avoid that.

H3: Their arguments are based on economics. Their perception of
inflation leads to their war-weariness.

H4: It is caused by polarization. They are only against what Biden
supports. (Iyengar et al.~2019; Pierson and Schickler 2020)

H5: They are misled by Russian bots.(Stukal wt al.~2019; Gorwa and
Guilbeault 2020; Hagen et al.~2022)

I must clarify that the research does not aim to test whether their
arguments are true or not, but how these beliefs interact with
conservative individuals' attitudes toward supporting Ukraine.

\hypertarget{variables-draft}{%
\subsubsection{Variables Draft}\label{variables-draft}}

DV: Whether a comment supports Ukraine

IV: The predicting factors in the said hypotheses

\hypertarget{data-of-interest}{%
\subsubsection{Data of Interest}\label{data-of-interest}}

H1-H4: Youtube comments(tuber)

H5: Twitter(kaggle)

\hypertarget{method-of-interest}{%
\subsubsection{Method of Interest}\label{method-of-interest}}

Text as Data H1-H4: Sentiment Analysis H5: Social Network

First, I will use the Tuber package to extract Youtube comments from
conservative media.

Then I will process these comments and tokenize them to create a
document by term matirx.

After that, I will try some word frequency analysis or clustering to
find out any features I may have ignored. If there are any, I will
iterate my hypotheses.

Finally, I will do sentiment analysis for the first four hypotheses, and
maybe social network analysis for hypothesis five, because I need
retweets to show the relationship between the Russian bots and the
retweeters.

\hypertarget{bibliography}{%
\subsubsection{Bibliography}\label{bibliography}}

Cha, Taesuh. 2016. ``The Return of Jacksonianism: The International
Implications of the Trump Phenomenon.'' \emph{The Washington Quarterly}
39(4): 83--97.

Clark, David H., Benjamin O. Fordham, and Timothy Nordstrom. 2016.
``Political Party and Presidential Decisions to Use Force: Explaining a
Puzzling Nonfinding: POLITICAL PARTY AND PRESIDENTIAL DECISIONS TO USE
FORCE.'' \emph{Presidential Studies Quarterly} 46(4): 791--807.

Clarke, Michael, and Anthony Ricketts. 2017. ``Donald Trump and American
Foreign Policy: The Return of the Jacksonian Tradition.''
\emph{Comparative Strategy} 36(4): 366--79.

George, Justin, and Todd Sandler. 2022. ``NATO Defense Demand, Free
Riding, and the Russo-Ukrainian War in 2022.'' \emph{Journal of
Industrial and Business Economics} 49(4): 783--806.

Gorwa, Robert, and Douglas Guilbeault. 2020. ``Unpacking the Social
Media Bot: A Typology to Guide Research and Policy.'' \emph{Policy \&
Internet} 12(2): 225--48.

Hagen, Loni et al.~2022. ``Rise of the Machines? Examining the Influence
of Social Bots on a Political Discussion Network.'' \emph{Social Science
Computer Review} 40(2): 264--87.

Iyengar, Shanto et al.~2019. ``The Origins and Consequences of Affective
Polarization in the United States.'' \emph{Annual Review of Political
Science} 22(1): 129--46.

Pierson, Paul, and Eric Schickler. 2020. ``Madison's Constitution Under
Stress: A Developmental Analysis of Political Polarization.''
\emph{Annual Review of Political Science} 23(1): 37--58.

Rolf, Jan Niklas. 2021. ``Donald Trump's Jacksonian and Jeffersonian
Foreign Policy.'' \emph{Policy Studies} 42(5--6): 662--81.

Sides, John, Michael Tesler, and Lynn Vavreck. 2018. ``Hunting Where the
Ducks Are: Activating Support for Donald Trump in the 2016 Republican
Primary.'' \emph{Journal of Elections, Public Opinion and Parties}
28(2): 135--56.

Stukal, Denis, Sergey Sanovich, Joshua A. Tucker, and Richard Bonneau.
2019. ``For Whom the Bot Tolls: A Neural Networks Approach to Measuring
Political Orientation of Twitter Bots in Russia.'' \emph{SAGE Open}
9(2): 215824401982771.

Uscinski, Joseph E. et al.~2021. ``American Politics in Two Dimensions:
Partisan and Ideological Identities versus Anti‐Establishment
Orientations.'' \emph{American Journal of Political Science} 65(4):
877--95.

Business Insider. 2022. ``Trump offers to lead group to mediate peace
with Russia.''
\url{https://www.businessinsider.com/trump-offers-lead-group-to-mediate-peace-russia-2022-9}
(accessed June 7, 2023).

Business Insider. 2023. ``Fox News cuts Trump, let Russia have bits of
Ukraine.''
\url{https://www.businessinsider.com/fox-news-cuts-trump-let-russia-have-bits-of-ukraine-2023-3}
(accessed June 7, 2023).

The Economist. 2022. ``Which countries have pledged the most support to
Ukraine.''
\url{https://www.economist.com/graphic-detail/2022/05/02/which-countries-have-pledged-the-most-support-to-ukraine}
(accessed June 7, 2023).

The Guardian. 2023. ``Boris Johnson meeting Donald Trump on Ukraine
during US tour.''
\url{https://www.theguardian.com/politics/2023/may/26/boris-johnson-meeting-donald-trump-ukraine-us-tour}
(accessed June 7, 2023).

The Hill. 2023. ``GOP lawmakers urge Biden to stop sending unrestrained
aid, weapons to Ukraine.''
\url{https://thehill.com/homenews/house/3961754-gop-lawmakers-urge-biden-to-stop-sending-unrestrained-aid-weapons-to-ukraine/}
(accessed June 7, 2023).

Reuters. 2023. ``Italy's Meloni ready to risk unpopularity over support
for Ukraine.''
\url{https://www.reuters.com/world/europe/italys-meloni-ready-risk-unpopularity-over-support-ukraine-2023-03-21/}
(accessed June 7, 2023).

\end{document}
